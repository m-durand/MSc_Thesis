\documentclass[twoside,openright,numbers,spanish]{ezthesis}
%% # Opciones disponibles para el documento #
%%
%% Las opciones con un (*) son las opciones predeterminadas.

%%
%% Formato de las referencias bibliogr'aficas:
%%   numbers          - numeradas, p.e. [1]
%%   authoryear (*)   - por autor y a'no, p.e. (Newton, 1997)
%%
%% Opciones adicionales:
%%   spanish         - tesis escrita en espa'nol
%%
%% Desactivar opciones especiales:
%%   nobibtoc   - no incluir la bibiolgraf'ia en el 'Indice general
%%   nofancyhdr - no incluir "fancyhdr" para producir los encabezados
%%   nocolors   - no incluir "xcolor" para producir ligas con colores
%%   nographicx - no incluir "graphicx" para insertar gr'aficos
%%   nonatbib   - no incluir "natbib" para administrar la bibliograf'ia

%% Paquetes adicionales requeridos se pueden agregar tambi'en aqu'i.
%% Por ejemplo:
%\usepackage{subfig}
%\usepackage{multirow}
\usepackage[spanish,activeacute]{babel}
\usepackage[utf8]{inputenc}
\usepackage{amsmath}
\usepackage{amsthm}
\usepackage{amssymb}
\usepackage{wrapfig}
\usepackage{minted}
\usemintedstyle{colorful}

%% # Datos del documento #
%% Nota que los acentos se deben escribir: \'a, \'e, \'i, etc.
%% La letra n con tilde es: \~n.

\author{Mar\'ia Fernanda Alcal\'a Durand}
\title{Aprendizaje Reforzado para el Juego de Distribuci\'on de Cerveza}
\degree{Maestra en Ciencia de Datos}
\supervisor{Dr. Adolfo Javier de Un\'anue Tiscare\~no}
\institution{Instituto Tecnol\'ogico Aut\'onomo de M\'exico}
\faculty{Divisi'on de Actuar\'ia, Estad\'istica y Matem\'aticas}
\department{Departamento Acad'emico de Matem\'aticas}

%% # M'argenes del documento #
%% 
%% Quitar el comentario en la siguiente linea para austar los m'argenes del
%% documento. Leer la documentaci'on de "geometry" para m'as informaci'on.

%\geometry{top=40mm,bottom=33mm,inner=40mm,outer=25mm}

%% El siguiente comando agrega ligas activas en el documento para las
%% referencias cruzadas y citas bibliogr'aficas. Tiene que ser *la 'ultima*
%% instrucci'on antes de \begin{document}.
\hyperlinking
\begin{document}

\graphicspath{{figs/}}

%% # Portada de la tesis #
\include{PaginaTitulo}

%% # Prefacios #
\preface
\include{Gracias}

%% # 'Indices y listas de contenido #
%% Quitar los comentarios en las lineas siguientes para obtener listas de
%% figuras y cuadros/tablas.
\tableofcontents
%\listoffigures
%\listoftables

%% # Cap'itulos #
\cuerpo
\chapter{Introducci'on}

\textit{Necesito una cita cool para empezar mi tesis.}
\begin{flushright}
 Fleo
 \end{flushright}

\vspace{10 pt}


%business dynamics
%sistemas complejos adaptativos
%poner ejemplos?

%dynamic stability : edge of chaos. no es que sean caóticos sino que la mayor parte de las fluctuaciones pequeñas se las comen los feedback, pero la línea de qué es "pequeño" no es nada clara
%ley de Ashby: para poder controlar algo, se necesita al menos igual nivel de complejidad


Uno de las principales dificultades de las cadenas de suministro es que los agentes encargados de optimizar las estrategias solamente pueden tomar decisiones "dentro" del eslabón en el que se encuentran, y no tienen información más allá de los eslabones inmediatemente conectados. Así, la información acerca de la demanda del consumidor se va diluyendo en cada nivel, además de que las decisiones tomadas tienen repercusiones más allá del futuro inmediato. \\

Los agentes optimizadores deben tratar de inferir el patrón global por medio de información local bastante restringida. Sin embargo, los datos que reciben obedecen al tiempo real y no tienen la oportunidad de repetir experimentos.\\

Un modelo computacional que se comporte suficientemente parecido al mundo real, en el que todos los demás eslabones tomen estrategias que también maximizarían sus beneficios podría dar una opción: el experimento es replicable tantas veces como sea necesario y cada eslabón puede conocer una extrategia óptima para una gran cantidad de demandas de consumidor posibles.\\

En este trabajo se modelará el Problema de Distribución de Cerveza, \textit{The Beer Distribution Game}, planteado por primera vez en la Escuela de Administraci\'on y Direcci\'on de Empresas Sloan del MIT en los años 60\footnote{En este momento no cuento con la fuente original.}, \\

Este documento tiene un formato simple y una estructura de propuesta de proyecto a propósito, dado que se pretende continuar trabajando en este proyecto hasta concretar una Tesis para obtener el grado de Maestría en Ciencia de Datos.
\include{AprendizajeReforzado}
\chapter{El Problema: Juego de Distribuci\'on de la Cerveza}

%The purpose of the game is to understand the distribution side dynamics of a multi-echelon supply chain used to distribute a single item, in this case, cases of beer.


%There is a one-point cost for holding excess inventory and a one-point cost for any backlog (old backlog + orders - current inventory).

%The game is used to illustrate one of the links between System Dynamics theory and the Feedback Control Theory which inspired it - that systems with positive feedback loops and high gain can lead to oscillation and overload,

La estructura se puede observar en la figura \ref{diagram_wikipedia}.\footnote{Imagen tomada de la página de Wikipedia \textit{The Beer Distribution Game}, bajo la licencia Creative Commons Attribution-Share Alike 3.0 Unported}\\


\begin{figure}[h]
\caption{Distribución de Cerveza}
\label{diagram_wikipedia}
\includegraphics[width=8cm]{Diagrama_Wikipedia.JPG}
\centering
\end{figure}

Las variables que tienen efecto en este problema son:
\begin{itemize}
    \item Demanda del Consumidor
    \item Tiempo de Ajuste de Inventario
    \item Tiempo de Envío
    \item Tiempo de Producción
\end{itemize}

Para cada uno de los agentes: tiendas minorista, mayorista y de distribución, y fábrica.\\

Este problema se ha estudiado antes por \citet{Strozzi}, por medio de Algoritmos Genéticos y por \citet{Chaharsooghi} por medio de $Q-learning$. Ambas metodolog\'ias \\

El aporte de este trabajo será agregar un componente de estacionalidad en el proveedor de la fábrica: el campo.

\subsection{\textit{Efecto Látigo}}

El \textit{Efecto Látigo} se ejemplifica con el siguiente escenario:


\begin{enumerate}
    \item El comprador, que generalmente compra $6$ cervezas, ahora quiere $10$, pero la tienda minorista solamente cuenta con $7$. El minorista le venderá todo su inventario, pues es la acci\'on que maximiza su ganancia. Debe decidir si volverá a tener un inventario de $6$ o si debe pedir un número mayor de cervezas, atendiendo la aparentemente creciente demanda. Decide pedir $9$ cervezas al siguiente nivel, la tienda de mayoreo.
    \item El mayorista cuenta con $17$ cervezas. Llena el pedido del minorista, pero decide que ten\'ia guardado demasiado inventario, as\'i que se queda con $8$ cervezas en su almac\'en, sin hacer una orden al siguiente nivel, la tienda de distribución.
    \item La tienda de distribuci\'on decide comprar $1$ unidad
\end{enumerate}

En este escenario, el mayorista obtuvo informaci\'on distorsionada acerca del repentino crecimiento en la demanda del comprador, mientras que la tienda de distribución . Si este comportamiento se mantiene durante algunos periodos más, recibiría la noticia (por medio de un incremento en las órdenes regulares) con un retraso considerable.\\

El \textit{Efecto Látigo} se refiere precisamente a este fenómeno: mientras más arriba en la cadena de suministro se encuentre un agente (es decir, más lejos del contacto directo con el comprador), más distorsionada es la información que tiene acerca de la verdadera demanda del consumidor.
\include{Conclusiones}

\appendix
%% Cap'itulos incluidos despues del comando \appendix aparecen como ap'endices
%% de la tesis.
\chapter{Ap\'endice}

\section{Escenarios de \textit{policy iteration}}

En este apartado se recopilan escenarios probados para el algoritmo \textit{Policy Iteration}. Tales escenarios constituyen, en su mayor\'ia, cambios en las condiciones iniciales del mundo, para visualizar los efectos en las estrategias aprendidas.\\

El escenario A muestra los resultados al iniciar uno de los agentes, en este caso el almac\'en regional, con un inventario de $500$ unidades, mientras que los dem\'as comienzan con $10$ unidades. Se puede observar que los agentes inferiores al almac\'en reconocen que hay existencias, y entonces su cantidad demandada es mayor a cero hasta el momento en el que el inventario del almac\'en se termina. Por otro lado, ni el almac\'en ni la f\'abrica presentan demanda constantemente positiva al principio del a\~no, pues ellos no pueden tomar decisiones diferentes en ese periodo debido la inyecci\'on de inventario.\\

Para el escenario B, se utiliz\'o una tendencia de producci\'on en los campos diferente de la utilizada en todo el trabajo, para demostrar que los agentes aprender\'an nuevas tendencias en caso de que estas cambien. Los nuevos valores fueron creados manualmente sin ninguna l\'ogica espec\'ifica. Puede notarse que todos los agentes cambian sus pol\'iticas \'optimas para comprar cuando hay producci\'on en los campos. \\

\begin{figure}[H]
\caption{Escenario A}
\label{scen_wholesale_500}
\includegraphics[width=9cm]{tesis_tex/figs/policyiteration_scen_wholesale500.png}
\centering
\end{figure}

\begin{figure}[H]
\caption{Escenario B}
\label{scen_alternative_supply}
\includegraphics[width=9cm]{tesis_tex/figs/policyiteration_scen_alternativesupply.png}
\centering
\end{figure}
%%\include{ApendiceB}

%% Incluir la bibliograf'ia. 
\bibliography{Bibliografia}

\end{document}




%%%%%%%%%%%%%%%%%%%%%%%%%% borrar en version final
 % The \cite command functions as follows:
 %   \citet{key} ==>>                Jones et al. (1990)
 %   \citet*{key} ==>>               Jones, Baker, and Smith (1990)
 %   \citep{key} ==>>                (Jones et al., 1990)
 %   \citep*{key} ==>>               (Jones, Baker, and Smith, 1990)
 %   \citep[chap. 2]{key} ==>>       (Jones et al., 1990, chap. 2)
 %   \citep[e.g.][]{key} ==>>        (e.g. Jones et al., 1990)
 %   \citep[e.g.][p. 32]{key} ==>>   (e.g. Jones et al., p. 32)
 %   \citeauthor{key} ==>>           Jones et al.
 %   \citeauthor*{key} ==>>          Jones, Baker, and Smith
 %   \citeyear{key} ==>>             1990
 %---------------------------------------------------------------------
% 
% \begin{center}
%\includegraphics[scale=0.5]{REER1970_Niveles.jpeg}
%\end{center}
% 
% 
% 